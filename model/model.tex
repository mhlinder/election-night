
\documentclass{article}

\usepackage[utf8]{inputenc}
\usepackage[T1]{fontenc}

\usepackage{lmodern}
\usepackage{pdflscape}

\usepackage{amssymb,amsmath}

\usepackage{bm}

\usepackage{graphicx}

\usepackage{longtable,booktabs}

\usepackage[unicode=true]{hyperref}
\hypersetup{colorlinks=true,
  linkcolor=black,
  citecolor=black,
  urlcolor=black,
  breaklinks=true}
\renewcommand{\href}[2]{#2\footnote{\url{#1}}}

\usepackage{apacite}
\usepackage[nottoc]{tocbibind}

\title{}
\author{Henry Linder \\ \texttt{mhlinder@gmail.com}}
\date{\today}

\begin{document}
{
  \centering
  \Large Election Forecasting Model \\[1em]
  \normalsize Henry Linder \\
  \texttt{mhlinder@gmail.com} \\
  \today \\[2em]
  \par
}

Notation derives from \cite{cargnoni1997bayesian}, itself citing
\cite{west1997}.

\section{Setup}

Denote by $\mathbf{p}_{i}=(p_{i1}, \dots,p_{i(r+1)})'$ a vector of $r$
observed proportions on day $t_{i}$, $i=1,\dots,n$,
$t_{i}\in\{0,\dots,T\}$. $\mathbf{p}_{i}$ gives voter responses to an
opinion poll about their preference among $r$ candidates, as well as
an additional category for ``Other''. Accompanying $\mathbf{p}_{t}$ is
a sample size, $N_{i}$, so that an estimate of the number of
respondents who prefer the $j$\textsuperscript{th} candidate may be
calculated as $y_{ij}=p_{ij}N_{i}$, $j=1,\dots,r$. The data is
reported in proportions, so to ensure the proper sample size after
calculating $y_{ij}$, we define $y_{i(r+1)} = N_{i} -
\sum_{j=1}^{r}y_{ij}$.

We consider the vector of counts
$\mathbf{y}_{i}=(y_{i1},\dots,y_{i(r+1)}$, and we consider a
multinomial likelihood:

\begin{gather}
  \left[ \mathbf{y}_{i}|\bm\pi_{t_{i}}, N_{i} \right]\sim
  \text{Mult.}(\bm\pi_{t_{i}}, N_{i}) \\
  \bm\eta_{i} = h(\bm\pi_{t_{i}}) = (h_{1}(\pi_{t_{i}1}), \dots,
  h_{r}(\pi_{t_{i}r}))'\in\mathbb{R}^{r}, \quad i=1,\dots, N. \\
  h_{j}(\pi_{tj}) = \log\frac{\pi_{tj}}{\pi_{t(r+1)}}, \quad
  j=1,\dots, r \\
  \bm\pi_{t_{i}} = h^{-1}(\bm\eta_{i}) = (h_{1}^{-1}(\bm\eta_{i}), \dots,
  h_{r+1}^{-1}(\bm\eta_{i}))' \in [0,1]^{r} \\
  h_{j}^{-1}(\bm\eta_{i}) =
  \frac{e^{\eta_{ij}}}{1+\sum_{j'=1}^{r}e^{\eta_{ij'}}}, \quad
  j=1,\dots, r\\
  h_{r+1}^{-1}(\bm\eta_{i}) = \frac{1}{1+\sum_{j'=1}^{r}e^{\eta_{ij'}}}
\end{gather}

where $\sum_{j=1}^{r+1}\pi_{tj}=1$.

We model $\bm\eta_{i}$ using a dynamic linear model, which provides a
structure to relate all polls that occur on the same day, i.e., the
sets $\{\mathbf{y}_{i} | t_{i}=t \}$, $t=1,\dots,T$.

\begin{gather}
  \bm\eta_{t} = \bm\beta_{t} + \bm\nu_{t},\quad \bm\nu_{t}\sim N_{r}(\bm 0,
  v\mathbf{I}_{r}) \\
  \bm\beta_{t} = \bm\beta_{t-1} + \bm\omega_{t}, \quad
  \bm\omega_{t}\sim N_{r}(\bm 0, w_{t}\mathbf{I}_{r})
\end{gather}

and we assume
$\bm\beta_{0}$ known. This model is a random walk for the daily
multinomial proportion vectors.

In the notation and terminology of \cite{west1997}, this is a constant
dynamic linear model where

\begin{gather}
  \mathbf{F} = \mathbf{I}_{r} \\
  \mathbf{G} = \mathbf{I}_{r}
\end{gather}

To construct a Bayesian model, we assign
constant variance in the observation equation ($v$), and 
iid time-varying variances in the system equation
($w_{t}$).

By the forward-filtering, backwards-sampling (FFBS) algorithm, the
prior distributions for all $\bm\beta_{t}$ depend recursively only on
$\bm\beta_{0}$, so we complete the model by assigning priors to the
parameters $\{ \bm\beta_{0}, v, w_{1},\dots, w_{T}\}$.

\begin{gather}
  \bm\beta_{0}\sim N_{r}(\mathbf{m}_{0}, C_{0}\mathbf{I}_{r}) \\
  v\sim \text{Inv.-Gamma}(\alpha_{0},\beta_{0}), \quad t=1,\dots,T \\
  w_{t} \sim \text{Inv.-Gamma}(\alpha_{0},\beta_{0})
\end{gather}

We choose diffuse priors for $v$, $w_{t}$, with
$\alpha_{0}=\beta_{0}=0.001$.

We specify the mean structure of the model with the forward-filtering,
backward-sampling procedure.

\subsection{Forward-filtering}

\cite{west1997} give the updates for the multivariate dynamic
linear model on page 582.

We suppose $\mathbf{m}_{0},C_{0}$ known, so $\bm\beta_{0}\sim
N_{r}(\mathbf{m}_{0},C_{0}\mathbf{I}_{r})$.

For $t=1,\dots,T$,

\begin{enumerate}
\item \textbf{Posterior at $t-1$}
  \begin{gather}
  \bm\beta_{t-1}|{\cal D}_{t-1} \sim N_{r}(\mathbf{m}_{t-1}, C_{t-1}\mathbf{I}_{r})
\end{gather}

\item \textbf{Prior at $t$}
  \begin{gather}
  \bm\beta_{t}|{\cal D}_{t-1} \sim N_{r}(\mathbf{m}_{t-1},
  R_{t}\mathbf{I}_{r})
\end{gather}

\item \textbf{One-step forecast}
  \begin{gather}
    \bm\eta_{t} | {\cal D}_{t-1} \sim N_{r}(\mathbf{m}_{t-1}, Q_{t}
    \mathbf{I}_{r})
\end{gather}
\item \textbf{Posterior at $t$}
  \begin{gather} \bm\beta_{t}|{\cal D}_{t} \sim N_{r}(\mathbf{m}_{t}, C_{t}) \end{gather}
\end{enumerate}

\begin{gather}
  \mathbf{m}_{t} = \mathbf{m}_{t-1} + A_{t} (\bm\eta_{t} -
  \mathbf{m}_{t-1}) \\
  R_{t} = C_{t-1} + w_{t} \\
  Q_{t} = (R_{t} + v) \\
  A_{t} = R_{t} / (R_{t} + v) \\
  C_{t} = R_{t} - A_{t}^{2}Q_{t}
\end{gather}

When filtering, also calculate $B_{t}$ given below.

\subsubsection{Missing observations}

Most series are comprised of unequally spaced time points, that is,
there may be ``missing observations''. In this case, we
have no data, but the state vector is not of interest: by properties
of a normal distribution  we can ``solve out'' the intervening time
periods. Practically speaking, this means the variance increases, but
the mean remains the same:

\begin{gather}
  \mathbf{m}_{t}=m_{t-1} \\
  R_{t} = C_{t-1} + w_{t}
\end{gather}

\subsection{Backward-sampling}

As written on page 570 of \cite{west1997}, we can then
sample backwards by the procedure procedure is then:

\begin{itemize}
\item Sample $\bm\beta_{T}|{\cal D}_{T} \sim N_{r}(\mathbf{m}_{T}, C_{T}\mathbf{I}_{r})$
\item For $t=n-1,n-2,\dots,1,0$, sample
  \begin{gather}
  \bm\beta_{t}|\bm\beta_{t+1},{\cal D}_{t} \sim N_{r}(\mathbf{h}_{t},
  H_{t}\mathbf{I}_{r}) \\
  \mathbf{h}_{t} = \mathbf{m}_{t} + B_{t}
  (\bm\beta_{t+1}-\mathbf{m}_{t+1}) \\
  H_{t} = C_{t} - B_{t}^{2}R_{t+1} \\
  B_{t} = \frac{C_{t}}{R_{t+1}}
\end{gather}
\end{itemize}

\bibliographystyle{apacite}
\bibliography{ref.bib}

\newpage

\begin{landscape}

  \section{Example dataset}

  This dataset was scraped for informational and research purposes from 
  \href{https://www.realclearpolitics.com/epolls/latest_polls/}{RealClearPolitics}.
  {\footnotesize
\begin{verbatim}
   Date            MoE Poll                           Year     N VoterType Start StartDate  End   EndDate    Abrams.D Kemp.R Other
   <chr>         <dbl> <chr>                         <int> <dbl> <chr>     <chr> <date>     <chr> <date>        <dbl>  <dbl> <dbl>
 1 10/29 - 10/31   3.7 Emerson                        2018   724 LV        10/29 2018-10-29 10/31 2018-10-31     0.47   0.49  0.04
 2 10/21 - 10/30   3   Atlanta Journal-Constitution   2018  1091 LV        10/21 2018-10-21 10/30 2018-10-30     0.47   0.47  0.06
 3 10/28 - 10/29   3.9 FOX 5 Atlanta/Opinion Savvy*   2018   623 LV        10/28 2018-10-28 10/29 2018-10-29     0.48   0.47  0.05
 4 10/21 - 10/22   3.4 FOX 5 Atlanta/Opinion Savvy*   2018   824 LV        10/21 2018-10-21 10/22 2018-10-22     0.48   0.48  0.04
 5 10/14 - 10/18   4.8 NBC News/Marist                2018   554 LV        10/14 2018-10-14 10/18 2018-10-18     0.47   0.49  0.04
 6 9/30 - 10/9     2.8 Atlanta Journal-Constitution*  2018  1232 LV        9/30  2018-09-30 10/9  2018-10-09     0.46   0.48  0.06
 7 10/3 - 10/8     4.9 WXIA-TV/SurveyUSA              2018   655 LV        10/3  2018-10-03 10/8  2018-10-08     0.45   0.47  0.08
 8 10/1 - 10/1     3.2 Landmark Communications*       2018   964 LV        10/1  2018-10-01 10/1  2018-10-01     0.46   0.48  0.06
 9 8/26 - 9/4      3.1 Atlanta Journal-Constitution*  2018  1020 LV        8/26  2018-08-26 9/4   2018-09-04     0.45   0.45  0.1 
10 7/27 - 7/29     3.8 Gravis                         2018   650 LV        7/27  2018-07-27 7/29  2018-07-29     0.46   0.44  0.1 
11 7/15 - 7/19     4.3 WXIA-TV/SurveyUSA              2018  1199 LV        7/15  2018-07-15 7/19  2018-07-19     0.44   0.46  0.1 
\end{verbatim}
    }
\end{landscape}

\end{document}